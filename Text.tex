\documentclass[10pt,twoside,slovak,a4paper]{article}
\usepackage{graphicx}


\title{Vplyv hier na naše zdravie}
\author{Adriana Kováčová\\
	{\small Slovenská technická univerzita v Bratislave}\\
	{\small Fakulta informatiky a informačných technológií}\\
	{\small {xkovacovaa@stuba.sk}}
	}
		
\date{November 2022}

\begin{document} %body
\maketitle
\begin{abstract}
V dnešnej modernej dobe sú počítačové alebo mobilné hry každodennou súčasťou života mládeže. Za posledné roky sa stali extrémne populárne. Využívajú sa na zábavu a odreagovanie sa od reálneho sveta plného stresu a povinností. Pomáhajú nám zlepšiť reflexy, pamäť, niektoré reflexy, naučiť sa nový jazyk – prevažne angličtinu, či ako sa lepšie rozhodnúť. Bohužiaľ, tento výdobytok modernej doby so sebou neprináša iba pozitíva.
Digitálne hry môžu na náš život vplývať aj negatívne. Môžu spôsobiť závislosť, stratu ostatných záujmov, problémy v osobnom živote a ďalšie nie veľmi prívetivé symptómy. Tieto pozitívne či negatívne stránky si podrobne rozoberieme v nasledujúcom texte.

\end{abstract}
\end{document} %body