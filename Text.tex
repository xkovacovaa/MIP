\documentclass[10pt,twoside,slovak,a4paper]{article}
\usepackage{graphicx}
\usepackage{url}
\usepackage{hyperref}
\usepackage{cite}

\title{Vplyv hier na naše zdravie\thanks{Semestrálny projekt v predmete Metódy inžinierskej práce, ak. rok 2022/2023, vedenie: Ing. Fedor Lehocki, PhD.}}
\author{Adriana Kováčová\\
	{\small Slovenská technická univerzita v Bratislave}\\
	{\small Fakulta informatiky a informačných technológií}\\
	{\small {xkovacovaa@stuba.sk}}
	}
		
\date{November 2022}

\begin{document} %body
\maketitle


\begin{abstract}
V dnešnej modernej dobe sú počítačové alebo mobilné hry každodennou súčasťou života mládeže. Za posledné roky sa stali extrémne populárne. Využívajú sa na zábavu a odreagovanie sa od reálneho sveta plného stresu a povinností. Pomáhajú nám zlepšiť reflexy, pamäť, niektoré reflexy, naučiť sa nový jazyk – prevažne angličtinu, či ako sa lepšie rozhodnúť. Bohužiaľ, tento výdobytok modernej doby so sebou neprináša iba pozitíva.
Digitálne hry môžu na náš život vplývať aj negatívne. Môžu spôsobiť závislosť, stratu ostatných záujmov, problémy v osobnom živote a ďalšie nie veľmi prívetivé symptómy. Tieto pozitívne či negatívne stránky si podrobne rozoberieme v nasledujúcom texte.
\end{abstract}


\section{Úvod}

Za prvú počítačovú hru sa všeobecne považuje hra Spacewar! z toku 1962, ktorú bolo po prvýkrát možné hrať na viacerých počítačoch. 1. Táto hra inšpirovala ostatných programátorov k vytvoreniu nových počítačových hier a je považovaná za jednu z najviac dôležitých a vplyvných hier v skorej histórií videohier. Počítače, a s nimi zároveň aj hry, sa postupom času vyvíjali, zdokonaľovali a popularizovali. 

S týmto výdobytkom modernej doby môžeme spájať aj Generáciu Z, ktorú môžeme zaradiť medzi roky narodenia 1996 až 2009. Je to prvá generácia, ktorá nezažila svet pred príchodom internetu. Ako táto internetová generácia rástla, zvyšovala sa aj miera hrania hier. S tým sa spájajú aj dôsledky popisované nižšie.

\begin{figure}
\includegraphics[width=\linewidth]{graf.png}
\caption{Broadband Search - Worldwide Number of Gamers, 2015-2022. \cite{graph}}
\end{figure}


\section{Pozitívne dopady hrania hier}
Pri hraní hier sa dokážeme naučiť mnoho nových vecí aj keď to nevnímame.  Pri komunikácií v multiplayer hrách sa používa anglický jazyk, čo nás núti sa ho. Niektoré hry zahrňujú hlasové alebo textové kanály, vďaka ktorým si môžeme nájsť nových kamarátov a zlepšiť svoju sociálnu interakciu s ostatnými ľuďmi.


\subsection{Zlepšenie riešení problémov a logického myslenia}
Pomocou daného typu hier, napríklad puzzle, rozvíjame naše myslenie. Snažíme sa splniť daný level, aby sme mohli v hre pokračovať. Trénujeme si tým zároveň mozog, aby mohol čo najrýchlejšie prísť s kreatívnym alebo najlepším spôsobom na vyriešenie daného problému. Vedci dokázali, že akčné hry trénujú ľudí, aby dokázali spraviť správne rozhodnutie rýchlejšie. \cite{decision}


\subsection{Multi-tasking}
Hranie mnohokrát vyžaduje rýchlu a presnú koordináciu pohybu rúk s reakciami mozgu. Hráč musí naraz vnímať viacero vecí, ako napríklad kde sa nachádza on a ostatní hráči, jeho cieľ, stav života a ostatné faktory, ktoré sú dôležité v danej hre. Zároveň sa naučí aj narábať so zásobami, ktoré sú obmedzené a ako ich najlepšie využiť. Štúdia Torontskej univerzity dokázala, že hráči sa dokážu naučiť rýchlejšie a presnejšie novým senzomotorickým zručnostiam ako je jazda na bicykli. \cite{sensorimotor}


\section{Negatívne dopady hrania hier}

\subsection{Mentálne zdravie}

\subsection{Psychické zdravie}



\section{Ako zlepšiť naše zdravie}

\subsection{Časovač}

\subsection{VR Realita}


\section{Diskusia}


\section{Záver}



\bibliography{literatura}

\bibliographystyle{alpha}

\end{document} %body