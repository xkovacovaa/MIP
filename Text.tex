\documentclass[10pt,twoside,slovak,a4paper]{article}
\usepackage{graphicx}
\usepackage{url}
\usepackage{hyperref}
\usepackage{cite}

\title{Vplyv hier na naše zdravie\thanks{Semestrálny projekt v predmete Metódy inžinierskej práce, ak. rok 2022/2023, vedenie: Ing. Fedor Lehocki, PhD.}}
\author{Adriana Kováčová\\
	{\small Slovenská technická univerzita v Bratislave}\\
	{\small Fakulta informatiky a informačných technológií}\\
	{\small {xkovacovaa@stuba.sk}}
	}
		
\date{November 2022}

\begin{document} %body
\maketitle


\begin{abstract}
V dnešnej modernej dobe sú počítačové alebo mobilné hry každodennou súčasťou života mládeže. Za posledné roky sa stali extrémne populárne. Využívajú sa na zábavu a odreagovanie sa od reálneho sveta plného stresu a povinností. Pomáhajú nám zlepšiť reflexy, pamäť, niektoré reflexy, naučiť sa nový jazyk – prevažne angličtinu, či ako sa lepšie rozhodnúť. Bohužiaľ, tento výdobytok modernej doby so sebou neprináša iba pozitíva.
Digitálne hry môžu na náš život vplývať aj negatívne. Môžu spôsobiť závislosť, stratu ostatných záujmov, problémy v osobnom živote a ďalšie nie veľmi prívetivé symptómy. Tieto pozitívne či negatívne stránky si podrobne rozoberieme v nasledujúcom texte.
\end{abstract}

\section{Úvod}

Za prvú počítačovú hru sa všeobecne považuje hra Spacewar! z toku 1962, ktorú bolo po prvýkrát možné hrať na viacerých počítačoch. 1. Táto hra inšpirovala ostatných programátorov k vytvoreniu nových počítačových hier a je považovaná za jednu z najviac dôležitých a vplyvných hier v skorej histórií videohier. Počítače, a s nimi zároveň aj hry, sa postupom času vyvíjali, zdokonaľovali a popularizovali. 

S týmto výdobytkom modernej doby môžeme spájať aj Generáciu Z, ktorú môžeme zaradiť medzi roky narodenia 1996 až 2009. Je to prvá generácia, ktorá nezažila svet pred príchodom internetu. Ako táto internetová generácia rástla, zvyšovala sa aj miera hrania hier. S tým sa spájajú aj dôsledky popisované nižšie.


\section{Pozitívne dopady hrania hier}


\section{Negatívne dopady hrania hier}


\section{Ako zlepšiť naše zdravie}


\section{Diskusia}


\section{Záver}



\bibliography{Bibliography}
\bibliographystyle{alpha}

\end{document} %body